\section{Exercise 1}

\subsection{Question 1}
The most interesting property of Laplacian matrix to me will be property 5: If the graph has $k$ isolated connected components, the multiplicity of the the zero eigenvalue of the Laplacian matrix is $k$. This property is interesting because it shows the relationship between the number of isolated connected components and the multiplicity of the zero eigenvalue of the Laplacian matrix. It so unbelievable that a geometric property of the graph can be represented by the algebraic property of the Laplacian matrix. It remind me the surprising feeling when I first learn about the fundamental group.

\subsection{Question 2}
We list some applications of Laplacian matrix mentioned in the slide Part 3:
\begin{enumerate}
    \item Spectral clustering: The Laplacian matrix can be used to partition the graph. Then we could use the partition to do spectral clustering.
    \item Semi-supervised learning: The Laplacian matrix can be used to do graph-based semi-supervised learning. We could use the Laplacian matrix to propagate the label of the labeled data to the unlabeled data. 
    \item Graph Neural Networks: The Laplacian matrix can be used to define the convolution operation in the graph neural networks.
\end{enumerate} 