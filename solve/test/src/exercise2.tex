
\section{Exercise 2}

\subsection{Programming tasks 1}
\subsubsection{Task 1}
We choose RBF kernel, the marginal likelihood and corresponding hyperparameters are shown in Table \ref{tab:ex2_1_1} as follows:
\begin{table}[H]
    \centering
    \begin{tabular}{|l|c|c|c|c|c|}
        \toprule
        \diagbox{variance}{lengthscale} & 0.5 & 0.75 & 1.0 & 1.25 & 1.5 \\
        \midrule
        0.5 & 3.629e-56 & 6.924e-54 & 1.051e-52 & 3.786e-52 & 5.532e-52 \\
        0.75 & 1.574e-55 & 2.812e-53 & 4.264e-52 & 1.637e-51 & 2.700e-51 \\
        1.0 & 1.617e-55 & 3.255e-53 & 5.526e-52 & 2.395e-51 & 4.511e-51 \\
        1.25 & 9.480e-56 & 2.341e-53 & 4.645e-52 & 2.308e-51 & 4.934e-51 \\
        1.5 & 4.366e-56 & 1.375e-53 & 3.237e-52 & 1.848e-51 & 4.443e-51 \\
        \bottomrule
    \end{tabular}
    \caption{Marginal likelihood and hyperparameters for RBF kernel}
    \label{tab:ex2_1_1}
\end{table}

\subsubsection{Task 2}

Use \text{optimize} function to find the optimal hyperparameters, the maximal marginal likelihood is 9.170e-23, and the corresponding hyperparameters are $variable = 1.28$, $\ell = 1.17$.

\subsubsection{Task 3}
Plot the posterior mean function and posterior standard deviation of the fitted GP using the function plot\_gp() from the tutorial.\ref{fig:ex2_1_3}

\begin{figure}[H]
    \centering
    \includegraphics[width=0.8\textwidth]{../img/task3.png}
    \caption{Posterior mean function and posterior standard deviation}
    \label{fig:ex2_1_3}
\end{figure}

\subsubsection{Task 4}

The predictive mean and standard deviation at the test points are shown in Table \ref{tab:ex2_1_4} as follows:

\begin{table}[H]
    \centering
    \begin{tabular}{|l|c|c|}
        \toprule
        X & posterior mean & posterior standard deviation \\
        \midrule
        12 & -1.012e+00 & 1.302e-01 \\
        22 & -1.709e-01 & 1.076e+00 \\
        \bottomrule
    \end{tabular}
    \caption{Predictive mean and variance at the test points}
    \label{tab:ex2_1_4}
\end{table}

We can find that the standard deviation at $X=22$ is much larger than that at $X=12$. That's because the range of training data is $[0, 20]$ and $X=22$ is out of the range, so the uncertainty is larger.