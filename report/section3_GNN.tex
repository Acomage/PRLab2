\section{使用图神经网络进行半监督分类}
在上一节中,我们使用了谱聚类和SBM模型对Mathlib4中的数学定理和证明进行了社群识别,并使用modularity、ARI和NMI指标对得到的社群结果进行了评价。从较低的ARI和NMI指标可以看出,我们社群发现的结果相较于按学科进行划分的结果有明显的差异;而从比学科划分更高的modularity指标来看,有理由认为社群发现的结果相较单纯的学科划分更能体现出数学概念之间的逻辑关系。

尽管谱聚类和SBM模型对于数据的社群发现可能提供了学科划分之外新的视角,但是两种方法作为无监督的学习方法,较低的可解释性和较大的评估难度是其固有的缺陷。

在本节中,我们将以每个节点的学科作为标签,使用图神经网络(GNNs)对Mathlib4中的数学定理和证明进行半监督分类。并将模型在测试集上的表现与按学科分类的真实结果进行对比。

\subsection{\texttt{PyTorch}数据准备}
我们使用\texttt{PyTorch Geometric}库来进行图卷积神经网络(GCNs)的构建和训练。为此,我们需要将Mathlib4中的数据转换为\texttt{PyTorch Geometric}中的\texttt{Data}对象,该对象包含了图的结构信息和节点特征信息。具体来说,包括:
\begin{itemize}
    \item \texttt{x}:节点特征矩阵,每一行代表一个节点的特征向量;
    \item \texttt{edge\_index}:边的索引矩阵,每一列代表一条边的起始节点和终止节点的索引;
    \item \texttt{y}:节点标签向量,每一个元素代表一个节点的标签;
    \item \texttt{train\_mask}:训练集掩码向量,每一个元素代表一个节点是否在训练集中(\texttt{val\_mask}和\texttt{test\_mask}同理)。
\end{itemize}

关于节点特征矩阵\texttt{x},因为我们分类仅仅用到的是图的依赖结构,节点特征并不包含额外的信息,因此我们将节点特征矩阵初始化为单位矩阵。值得注意的是,其他可以考虑的选择包括使用Laplacian矩阵的特征向量作为节点特征,或者使用预训练的embedding作为节点特征。

关于图的边索引矩阵\texttt{edge\_index},我们将Mathlib4中的依赖关系作为无向边处理。为了适应GCN的结构,这一简化处理损失了依赖关系的方向性,尽管在数学证明的范畴下这一简化可能是合理的。然而,可能的改进方向包括使用自然考虑到图的方向性的模型如Graph Attention Networks(GATs)。

关于节点标签向量\texttt{y},我们将每个节点的学科作为标签。由于\texttt{Torch}期望标签是从0开始的整数,我们将学科的字符串标签映射到整数标签。

关于训练-验证-测试集的划分,我们选取三者的比例为$8:1:1$,并使用掩码向量做出标记。

\subsection{构建GCN模型并训练}
我们使用\texttt{PyTorch Geometric}库中的\texttt{GCNConv}层来构建GCN模型。我们的模型初步包括一层GCN层后接一个ReLU激活函数,最后以线性层输出。我们使用交叉熵损失函数和Adam优化器进行训练。